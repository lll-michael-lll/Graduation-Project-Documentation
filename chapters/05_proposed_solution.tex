% ============================================================
% 5. PROPOSED SOLUTION
% ============================================================

\begin{sectionintro}{5}{Proposed Solution}{
  \begin{itemize}[leftmargin=1.5em]
    \item Zero Trust architecture and credential abstraction
    \item Dynamic proxy management
    \item JWT-based authentication and token lifecycle
    \item Policy enforcement and audit logging
    \item Cross-platform CLI with secure storage
  \end{itemize}
}
\lettrine[lines=3, lhang=0, loversize=0.15]{\color{primaryBlue}T}{he zGate solution addresses database security vulnerabilities through Zero Trust architecture.} This chapter presents our comprehensive approach to eliminating credential exposure through dynamic proxy management, JWT-based authentication, and real-time policy enforcement. We explore how the gateway server, CLI, and web dashboard integrate to provide identity-based access control while maintaining operational efficiency.
\end{sectionintro}

\section{Solution Overview}

\IEEEPARstart{T}{he} proposed solution is \textbf{zGate}, a comprehensive Zero-Trust Database Access Platform that fundamentally changes how organizations manage database security and access control. Rather than relying on shared credentials and static permissions, zGate introduces a dynamic, policy-driven approach where temporary credentials are created for each user session, eliminating the risks associated with credential sprawl and unauthorized access.

\subsection{The Core Problem}

Traditional database access models suffer from several critical security and operational challenges:

\begin{enumerate}
    \item \textbf{Shared Credentials:} Multiple users share common database accounts, making it impossible to audit who performed which actions
    \item \textbf{Static Permissions:} Once granted, permissions remain indefinitely, increasing the attack surface
    \item \textbf{Credential Sprawl:} Database passwords stored in configuration files, wikis, and password managers
    \item \textbf{Lack of Visibility:} No centralized view of who has access to which databases
    \item \textbf{Manual Overhead:} Granting and revoking access requires manual database administration
    \item \textbf{Compliance Gaps:} Difficult to prove least-privilege access and maintain audit trails
\end{enumerate}

\subsection{The zGate Solution}

zGate addresses these challenges through a multi-layered architecture built on three core principles:
\begin{figure}[H]
    \centering
    \includegraphics[width=0.8\textwidth]{images/zgate-sol.png}
    \caption{zgate proposed solution architecture}
    \label{fig:zgate-solution}
\end{figure}
\section{Core Components}

\subsection{zGate Server (Backend)}

The server is the heart of the platform, handling all security, policy enforcement, and proxy operations.

\subsubsection{Backend Layers}
\begin{figure}[H]
    \centering
    \includegraphics[width=0.8\textwidth]{images/backend.png}
    \caption{zGate backend architecture showing authentication, policy engine, and proxy layers}
    \label{fig:zgate-backend-architecture}
\end{figure}
\subsubsection{Key Features}

\textbf{1. JWT-Based Authentication}
\begin{itemize}
    \item Access tokens (15-minute TTL) for API requests
    \item Refresh tokens (7-day TTL) for session persistence
    \item Secure token generation with HMAC-SHA256 signatures
    \item Automatic token refresh in CLI
\end{itemize}

\textbf{2. Real-Time Policy Enforcement}
\begin{itemize}
    \item Permissions evaluated from current configuration on every request
    \item Role-based access control (RBAC) with support for multiple roles per user
    \item Custom permissions at user level for exceptions
    \item Immediate effect when permissions are changed (no cache invalidation needed)
\end{itemize}

\textbf{3. Dynamic Proxy Management}
\begin{itemize}
    \item On-demand creation of proxy sessions
    \item Automatic port allocation from ephemeral range
    \item Temporary database user creation with appropriate grants
    \item Automatic cleanup on disconnect or crash recovery
\end{itemize}

\textbf{4. Protocol Abstraction}
\begin{itemize}
    \item Pluggable architecture for database types
    \item Current support: MSSQL, MySQL
    \item Easy extensibility for PostgreSQL, Oracle, etc.
    \item Database-specific SQL for user management
\end{itemize}

\textbf{5. Comprehensive Logging}
\begin{itemize}
    \item Structured logging with key-value pairs
    \item Every security event logged with user context
    \item Support for JSON output for log aggregation
    \item Configurable log levels (DEBUG, INFO, WARN, ERROR)
\end{itemize}

\subsection{zGate CLI (Client)}

The command-line interface provides developers with a simple, secure way to access databases.

\subsubsection{CLI Workflow}
\begin{figure}[H]
    \centering
    \includegraphics[width=0.8\textwidth]{images/CLI-workflow.png}
    \caption{zGate CLI workflow illustrating token management and database access process}
    \label{fig:CLI-workflow}
\end{figure}
\subsubsection{CLI Features}

\textbf{1. Secure Token Storage}
\begin{itemize}
    \item OS-native keyring integration (Keychain on macOS, Credential Manager on Windows, Secret Service on Linux)
    \item Encrypted file fallback for headless environments
    \item Never stores plaintext credentials
\end{itemize}

\textbf{2. Automatic Token Refresh}
\begin{itemize}
    \item Detects token expiry proactively
    \item Refreshes access token using refresh token
    \item Seamless user experience without re-authentication
\end{itemize}

\textbf{3. User-Friendly Interface}
\begin{itemize}
    \item Simple, intuitive commands following Unix conventions
    \item Colored output for better readability
    \item Helpful error messages with troubleshooting hints
    \item Progress indicators for long operations
\end{itemize}

\textbf{4. Cross-Platform Support}
\begin{itemize}
    \item Single binary for Windows, macOS, Linux
    \item No runtime dependencies
    \item Portable and lightweight (\textasciitilde10MB)
\end{itemize}

\subsection{zGate WebUI (Admin Dashboard)}

The web interface provides administrators with a comprehensive control panel for managing the entire platform.

\subsubsection{Dashboard Architecture}

\subsubsection{Admin Features}

\textbf{1. Database Management}
\begin{itemize}
    \item Add/edit/delete database configurations
    \item Test database connectivity before saving
    \item View all configured databases with status
    \item Manage available permissions per database
\end{itemize}

\textbf{2. User \& Role Management}
\begin{itemize}
    \item Create users with bcrypt-hashed passwords
    \item Assign multiple roles to users
    \item Define roles with database-specific permissions
    \item Add custom permissions to individual users
\end{itemize}

\textbf{3. Session Monitoring}
\begin{itemize}
    \item Real-time view of active database connections
    \item See: user, database, port, temporary username, duration
    \item Terminate sessions if needed (security incidents)
    \item Filter and search sessions
\end{itemize}

\textbf{4. Audit Trail}
\begin{itemize}
    \item Comprehensive logs of all user activities
    \item Filter by user, date range, action type
    \item Search within log entries
    \item Export for compliance reporting
\end{itemize}

\textbf{5. Advanced Features}
\begin{itemize}
    \item Shared account pools for efficient credential management
    \item Personal database accounts for specific users
    \item Query execution interface (future)
    \item Connected databases view
\end{itemize}

\section{Key Technologies \& Design Decisions}

\subsubsection{Technology Stack}
\begin{figure}[H]
    \centering
    \includegraphics[width=0.8\textwidth]{images/tech-stack.png}
    \caption{zGate technology stack}
    \label{fig:technology-stack}
\end{figure}

\section{Security Architecture}

Security is woven into every layer of the zGate platform through multiple defense mechanisms.

\subsection{Security Layers}
\begin{figure}[H]
    \centering
    \includegraphics[width=0.8\textwidth]{images/security-layers.png}
    \caption{Multi-layered security architecture in zGate platform}
    \label{fig:security-layers}
\end{figure}
\subsection{Security Guarantees}

\begin{table}[h]
\centering
\resizebox{\textwidth}{!}{%
\begin{tabular}{|l|l|l|}
\hline
\textbf{Threat} & \textbf{Mitigation} & \textbf{Implementation} \\
\hline
Credential Theft & No static credentials exposed to users & Temporary users with auto-generated passwords \\
\hline
Privilege Escalation & Database-level permission enforcement & Grants limited to user's role (read/write/admin) \\
\hline
Unauthorized Access & Real-time policy evaluation & Every connection request checked against current config \\
\hline
Session Hijacking & Short-lived JWT tokens & 15-minute access token expiry \\
\hline
Password Cracking & Strong password hashing & bcrypt with work factor 10 \\
\hline
Audit Evasion & Mandatory logging & Every API call, connection, disconnect logged \\
\hline
Resource Exhaustion & Automatic cleanup & Temporary users deleted within 30 seconds of disconnect \\
\hline
Man-in-the-Middle & Encrypted connections & TLS support for backend database connections \\
\hline
\end{tabular}%
}
\caption{Security threats and their mitigations in zGate}
\label{tab:security-guarantees}
\end{table}

\section{Advantages of the Proposed Solution}

\subsubsection{Key Benefits}

\subsubsection{Enhanced Security}
\begin{itemize}
    \item Eliminates shared credentials
    \item Temporary users auto-deleted
    \item No credential sprawl
    \item Cryptographically secure random passwords
    \item Short-lived JWT tokens
    \item Database-level permission enforcement
\end{itemize}

\subsubsection{Operational Efficiency}
\begin{itemize}
    \item Automated user provisioning
    \item Self-service for developers (via CLI)
    \item Centralized access management
    \item No manual database administration
    \item Quick onboarding/offboarding
\end{itemize}

\subsubsection{Compliance \& Audit}
\begin{itemize}
    \item Complete audit trail
    \item User-attributed actions
    \item Provable least-privilege
    \item Real-time access reporting
    \item Exportable logs
\end{itemize}

\subsubsection{Developer Experience}
\begin{itemize}
    \item Simple CLI commands
    \item Works with existing tools
    \item No VPN/bastion required
    \item Automatic token refresh
    \item Cross-platform support
\end{itemize}

\subsubsection{Scalability}
\begin{itemize}
    \item Handles thousands of concurrent sessions
    \item Stateless API (horizontal scaling)
    \item Lightweight memory footprint
    \item Dynamic resource allocation
\end{itemize}

