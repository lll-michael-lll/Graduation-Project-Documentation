% ============================================================
% 19. ROADMAP FOR TERM 2
% ============================================================

\begin{sectionintro}{19}{Roadmap for Term 2}{
  \begin{itemize}[leftmargin=1.5em]
    \item Remaining feature implementations
    \item Enterprise identity management integration
    \item Observability and monitoring systems
    \item Security enhancements and hardening
    \item Validation and testing strategy
  \end{itemize}
}
\lettrine[lines=3, lhang=0.1, loversize=0.15]{\color{primaryBlue}T}{erm~2 represents the evolution from proof of concept} to production-ready system. This chapter outlines the roadmap for completing zGate's transformation into an enterprise-grade database security solution. Focus areas include integrating advanced identity management, implementing comprehensive observability features, and conducting rigorous validation to ensure reliability and security at scale.
\end{sectionintro}

\chapter{Roadmap for Term 2}

Term~2 represents the critical transition from a functional Proof of Concept to a production-ready database security solution. Building upon the foundational work completed in Term~1, the development team will focus on three primary areas: integrating enterprise-grade identity management, implementing comprehensive observability features, and conducting rigorous validation to ensure system reliability and security.

\section{Remaining Features}

The following features are scheduled for implementation during Term~2, organized by milestone and priority.

\subsection{External Identity Provider Integration (Milestone 4)}

The current POC utilizes an internal token-based authentication system. Term~2 will extend this to support federated identity management through external providers.

\begin{itemize}
    \item \textbf{OAuth2/OIDC Implementation:} Integrate industry-standard authentication protocols to enable Single Sign-On (SSO) capabilities.
    \item \textbf{Identity Provider Support:} Configure connectors for major identity providers including:
    \begin{itemize}
        \item Google Workspace
        \item Microsoft Azure Active Directory
        \item Okta
        \item Generic SAML 2.0 providers
    \end{itemize}
    \item \textbf{Claims-to-Role Mapping:} Develop a flexible mapping engine that translates IdP claims (groups, roles, attributes) to zGate's internal RBAC policies.
    \item \textbf{Just-in-Time Provisioning:} Implement automatic user provisioning based on IdP attributes upon first login.
\end{itemize}

\subsection{Enhanced Observability Features (Milestone 5)}

Transform the initial dashboard into a comprehensive monitoring and compliance platform.

\begin{itemize}
    \item \textbf{Real-Time Metrics Dashboard:}
    \begin{itemize}
        \item Connection pool utilization graphs
        \item Query throughput and latency percentiles
        \item Active session monitoring
        \item Database health indicators
    \end{itemize}
    \item \textbf{Audit Logging System:}
    \begin{itemize}
        \item Structured logging of all database access events
        \item Query content capture with optional redaction
        \item User activity timeline visualization
        \item Export capabilities for compliance reporting (CSV, JSON, SIEM integration)
    \end{itemize}
    \item \textbf{Alerting Framework:}
    \begin{itemize}
        \item Configurable threshold-based alerts
        \item Anomaly detection for unusual access patterns
        \item Integration with notification channels (email, Slack, webhooks)
    \end{itemize}
\end{itemize}

\subsection{Advanced Policy Engine Enhancements}

Extend the current policy engine with more granular control mechanisms.

\begin{itemize}
    \item \textbf{Time-Based Access Controls:} Implement policies that restrict database access to specific time windows.
    \item \textbf{Query Complexity Limits:} Add configurable limits on query complexity to prevent resource exhaustion.
    \item \textbf{Row-Level Security Policies:} Develop mechanisms to enforce row-level filtering based on user identity.
    \item \textbf{Dynamic Masking Rules:} Extend data masking to support context-aware redaction based on user roles.
\end{itemize}

%-----------------------------------------------------------

\section{Architecture Improvements}

Based on lessons learned during POC development, the following architectural enhancements are planned.

\subsection{Scalability Enhancements}

\begin{itemize}
    \item \textbf{Horizontal Scaling Support:} Implement stateless proxy instances that can be load-balanced for high availability.
    \item \textbf{Connection Multiplexing:} Optimize connection pooling to support higher concurrent user loads with fewer backend connections.
    \item \textbf{Caching Layer:} Introduce caching for policy decisions and session metadata to reduce latency.
\end{itemize}

\subsection{Security Hardening}

\begin{itemize}
    \item \textbf{TLS Certificate Management:} Implement automated certificate rotation and enhanced certificate validation.
    \item \textbf{Secret Management Integration:} Support for external secret stores (HashiCorp Vault, AWS Secrets Manager).
    \item \textbf{Audit Trail Integrity:} Implement cryptographic signing of audit logs to ensure tamper-evidence.
\end{itemize}

\subsection{Deployment Improvements}

\begin{itemize}
    \item \textbf{Container Orchestration:} Official Docker images and Kubernetes Helm charts for production deployment.
    \item \textbf{Configuration Management:} Migration to environment-based configuration with validation on startup.
    \item \textbf{Health Check Endpoints:} Enhanced liveness and readiness probes for orchestrator integration.
\end{itemize}

%-----------------------------------------------------------

\section{Performance Goals}

Table~\ref{tab:performance-goals} outlines the target performance metrics for the production release.

\begin{table}[H]
\centering
\caption{Term 2 Performance Targets}
\label{tab:performance-goals}
\begin{tabular}{|l|l|l|}
\hline
\textbf{Metric} & \textbf{Target} & \textbf{Measurement Method} \\
\hline
Latency Overhead (p50) & $<$ 2ms & Benchmarking with synthetic load \\
Latency Overhead (p99) & $<$ 10ms & Benchmarking under stress \\
Concurrent Connections & $\geq$ 1000 & Load testing with connection ramp \\
Connection Setup Time & $<$ 50ms & End-to-end connection timing \\
Memory Footprint & $<$ 500MB (idle) & Resource monitoring under load \\
CPU Utilization & $<$ 30\% (normal load) & Profiling during benchmark \\
\hline
\end{tabular}
\end{table}

\subsection{Optimization Strategies}

To achieve these performance targets, the following optimization strategies will be employed:

\begin{enumerate}
    \item \textbf{Profiling-Driven Optimization:} Use Go's built-in profiler (pprof) to identify and eliminate bottlenecks.
    \item \textbf{Memory Pool Management:} Implement buffer pooling to reduce garbage collection overhead.
    \item \textbf{Protocol Parser Optimization:} Optimize hot paths in protocol parsing using zero-copy techniques where applicable.
    \item \textbf{Goroutine Pool Management:} Implement worker pools to prevent goroutine explosion under high load.
\end{enumerate}

%-----------------------------------------------------------

\section{Testing \& Validation Plan}

A comprehensive testing strategy will ensure system reliability and security before final delivery.

\subsection{Automated Testing Framework}

\begin{itemize}
    \item \textbf{Unit Tests:} Target minimum 80\% code coverage for core proxy logic and security modules.
    \item \textbf{Integration Tests:} End-to-end tests covering authentication flows, query processing, and data masking across all supported databases.
    \item \textbf{Regression Test Suite:} Automated test suite executed on every code commit via CI/CD pipeline.
\end{itemize}

\subsection{Performance Validation}

\begin{itemize}
    \item \textbf{Load Testing:} Use tools such as \texttt{pgbench} (PostgreSQL), \texttt{mysqlslap} (MySQL), and custom scripts to simulate production workloads.
    \item \textbf{Stress Testing:} Evaluate system behavior under extreme conditions (connection floods, memory pressure).
    \item \textbf{Latency Profiling:} Measure and document latency distribution across different query types and database backends.
\end{itemize}

\subsection{Security Validation}

\begin{itemize}
    \item \textbf{Penetration Testing:} Internal security audit focusing on:
    \begin{itemize}
        \item SQL injection bypass attempts
        \item Authentication bypass vectors
        \item Token forging and replay attacks
        \item Policy engine circumvention
    \end{itemize}
    \item \textbf{Threat Modeling:} Document potential attack vectors and corresponding mitigations.
    \item \textbf{Dependency Audit:} Scan all dependencies for known vulnerabilities using automated tools.
\end{itemize}

\subsection{User Acceptance Testing}

\begin{itemize}
    \item \textbf{Supervisor Demonstrations:} Weekly demonstrations of new features to project supervisors.
    \item \textbf{Documentation Review:} Validation that all user-facing documentation accurately reflects system behavior.
    \item \textbf{Demo Environment:} Prepare a stable demonstration environment for final presentation.
\end{itemize}

\subsection{Validation Timeline}

Table~\ref{tab:testing-timeline} presents the planned testing phases and their timelines.

\begin{table}[H]
\centering
\caption{Testing \& Validation Timeline}
\label{tab:testing-timeline}
\begin{tabular}{|l|l|l|}
\hline
\textbf{Testing Phase} & \textbf{Timeline} & \textbf{Focus Area} \\
\hline
Unit \& Integration Testing & Ongoing (all sprints) & Continuous quality assurance \\
Performance Benchmarking & May 1--15 & Latency and throughput validation \\
Security Audit & May 15--31 & Penetration testing and hardening \\
User Acceptance Testing & June 1--10 & Feature validation and feedback \\
Final Documentation Review & June 10--15 & Documentation completeness \\
Demo Environment Preparation & June 15--20 & Final presentation preparation \\
\hline
\end{tabular}
\end{table}

%-----------------------------------------------------------

\section{Term 2 Development Timeline}

Figure~\ref{fig:term2-timeline} illustrates the project timeline for Term~2, depicting the progression through the remaining three milestones from February to June 2026.

\begin{figure}[htbp]
    \centering
    \includegraphics[width=\textwidth]{images/Term 2 Timeline.png}
    \caption{Term 2 Project Timeline (February -- June 2026)}
    \label{fig:term2-timeline}
\end{figure}

% Mermaid Gantt Chart Syntax for Term 2 Timeline:
% 
% ```mermaid
% gantt
%     title Term 2 Development Timeline (Feb - June 2026)
%     dateFormat  YYYY-MM-DD
%     axisFormat  %b %Y
%     
%     section Milestone 4
%     External IAM & Identity Integration     :m4_iam, 2026-02-01, 2026-02-28
%     SSO Integration (Google/Okta/Azure AD)  :m4_sso, 2026-02-01, 2026-02-14
%     OAuth2/OIDC Authentication Flow         :m4_oauth, 2026-02-08, 2026-02-21
%     RBAC Mapping & Policy Engine            :m4_rbac, 2026-02-15, 2026-02-28
%     
%     section Milestone 5
%     Observability & Production Readiness    :m5_obs, 2026-03-01, 2026-04-30
%     Metrics Visualization & Dashboard       :m5_metrics, 2026-03-01, 2026-03-21
%     Audit Logging Implementation            :m5_audit, 2026-03-15, 2026-04-07
%     Error Handling & System Robustness      :m5_error, 2026-04-01, 2026-04-30
%     
%     section Milestone 6
%     Final Validation & Delivery             :m6_final, 2026-05-01, 2026-06-15
%     Performance Benchmarking & Load Testing :m6_perf, 2026-05-01, 2026-05-21
%     Security Auditing & Penetration Testing :m6_sec, 2026-05-15, 2026-05-31
%     Final Documentation & Reports           :m6_doc, 2026-05-20, 2026-06-10
%     Demo Environment & Presentation Prep    :m6_demo, 2026-06-01, 2026-06-15
% ```
